\section{Linear equations}
\subsection{Linear and affine functions}
\textbullet Superposition condition:$f(\alpha x + \beta y) = \alpha f(x) + \beta f(y)$\\
\textbullet Such an f is called Linear\\
\textbf{Matrix vector product function}:\\
\textbullet A is $m \times n$ matrix such that $f(x)=Ax$\\
\textbullet f is linear:  $f(\alpha x + \beta y)=A(\alpha x + \beta y)=\alpha f(x) + \beta f(y)$\\
\textbullet Converse is true: If $f:R^{n} \mapsto R^{m}$ is linear, then \\
$f(x)=f(x_{1}e_{1}+x_{2}e_{2}+...x_{n}e_{n})$\\
$=x_{1}f(e_{1})+x_{2}f(e_{2})+...x_{n}f(e_{n})$
$=Ax$ with $A = [f(e_{1})+f(e_{2})+...f(e_{n})]$
\newline \textbf{Affine Functions}: $f:R^{n} \mapsto R^{m}$ is affine if it is a linear function plus a constant i.e $f(x)=Ax+b$ same as $f(\alpha x + \beta y)=\alpha f(x)+\beta f(y)$ holds for all x, y and $\alpha,\beta$ such that $\alpha+\beta=1$ 
\newline A and b can be calculated as \\
$A=[f(e_{1})-f(0)\thickspace f(e_{2})-f(0) ... f(e_{n})-f(0)]] ; $\\
$b= f(0)$\\
\textbullet Affine functions sometimes incorrectly called linear functions
\subsection{Linear function models}
Price elasticity of demand
$\delta_i^{price}=(p_i^{new}-p_{i})/p_{i}$: fractional changes in prices
$\delta_i^{dem}=(d_i^{new}-d_{i})/d_{i}$: fractional change in demand
Price demand elasticity model: $\delta^{dem}=E\delta^{price}$\\
\textbf{Taylor series approximation}\\
\textbullet The (first-order) Taylor approximation of f near (or at) the point z:\\
\begin{scriptsize}
$\hat{f}(x)= f(z) + \frac{\partial f}{\partial x_1}(z)(x_1 - z_1) + ... +  \frac{\partial f}{\partial x_n}(z)(x_n - z_n)$
\end{scriptsize}\\
\textbullet in compact notation:  \\$\hat{f}(x) = f (z) + Df (z)(x - z)$
\subsection{Systems of linear equations}
\textbullet set (or system) of m linear equations in n variables $x_1,.., x_n$: \\
$A_{11}x_1 + A_{12}x_2 + ... + A_{1n}x_n = b_1$\\
$A_{21}x_1 + A_{22}x_2 + ... + A_{2n}x_n = b_2$\\
.\\
.\\
$A_{m1}x_1 + A_{m2}x_2 + ... + A_{mn}x_n = b_m$\\

\textbullet \textbf{systems of linear equations classified as}\\
\thickspace -- under-determined if m < n (A wide)\\
\thickspace -- square if m = n (A square)\\
\thickspace -- over-determined if m > n (A tall)\\

\textbf{Balancing equation example}\\
\textbullet consider reaction with m types of atoms, p reactants, q products\\
\textbullet $m \times p$ reactant matrix R is defined by\\
$R_{ij}$ = number of atoms of type i in reactant $R_j$\\
for i = 1,..,m and j = 1,..,p\\
\textbullet with $a = (a_1,..,a_p)$ (vector of reactant coefficients)\\
$Ra$ = (vector of) total numbers of atoms of each type in reactants\\
\textbullet define product $m \times q$ matrix P in similar way\\
\textbullet m-vector $Pb$ is total numbers of atoms of each type in products\\
\textbullet conservation of mass is $Ra = Pb$
\textbullet  conservation of mass is\\
$[R - P][a\thickspace b]^T = 0$\\
\textbullet simple solution is $a = b = 0$\\
\textbullet to find a nonzero solution, set any coefficient (say, $a_1$) to be 1\\
\textbullet balancing chemical equations can be expressed as solving a set of m + 1 linear equations in p + q variables \\
$
\begin{bmatrix}
    R & -P \\
    e_1^T & 0
    
\end{bmatrix} 
\begin{bmatrix}
    a \\
    b
\end{bmatrix}=e_{m+1}
$\\
(we ignore here that $a_i$ and $b_i$ should be nonnegative integers)