\section{Matrices}
\subsection{Matrices}
The set of real m × n matrices is denoted $R^{m \times n}$
\subsection{Zero and identity matrices}
\textbullet \textit{Zero}: All elements equals 0.\\
\textbullet \textit{Identity}: All elements equals 0 and diagonal element equals 1.\\
\textbullet \textit{Sparse}: If many entries are 0\\
\textbullet \textit{Diagonal}: off-diagonal entries are zero\\
\textbullet \textit{Triangular}: \textit{upper triangular} if $A_{ij} = 0 \text{ for } i>j$, and it is \textit{lower triangular} if $A_{ij} =0 \text{ for } i<j$\\
\textbullet \textit{Adjacency Matrix}: \\
For, $R={(1,2), (1,3), (2,1), (2,4), (3,4), (4,1)}$\\
$
A_{ij} = \left\{
  \begin{array}{@{}ll@{}}
    1, & (i,j) \in R\\
    0, & (i, j) \not\in R
  \end{array}\right.
$\\
A relation R on ${1,...,n}$ is represented by the n×n matrix A with $A_{ij} = 1$, if there exists a edge else , $A_{ij} = 0$
\subsection{Transpose, addition and norm}
\textbf{Block matrix Transpose}\\
$
\begin{bmatrix}
    A & B \\
    C & D
    
\end{bmatrix}^{T}= 
\begin{bmatrix}
    A^T & C^T \\
    B^T & D^T
\end{bmatrix}
$\\
\textbf{Symmetric matrix}: $A = A^T$\\
\textbf{Properties of matrix addition}\\
\textbullet Commutativity: $A+B=B+A$\\
\textbullet Associativity: $(A + B) + C = A + (B + C)$\\
\textbullet Addition with zero matrix: $A+0=0+A=A$\\
\textbullet Transpose of sum:  $(A+B)^T = A^T +B^T$\\
If A is a matrix and $\beta$ , $\gamma$ are scalars
$(\beta + \gamma)A = \beta A + \gamma A, 
(\beta \gamma)A = \beta(\gamma A)
$\\
\textbf{Matrix norm}
$\Vert A \Vert = \sqrt{\sum_{i=1}^n\sum_{j=1}^m A^2_{ij}}$
matrix norm satisfies the properties of any norm

\subsection{Matrix-vector multiplication}
A is an $m\times n$ matrix and x is an n-vector, then the matrix-vector product $y = Ax$\\

$y_i = \sum_{k=1}^n A_{ik} x_k =A_{i1}x_1 +...+A_{in}x_n$ for $i=1...m$\\
\textbullet \textbf{Row and column interpretations.}\\
$y = Ax$ can be expressed as $y_i=b^T_ix, i=1,..,m$ where $b^T_1,...,b^T_m$ are rows of A\\
\textbullet $y = Ax$ could also be expressd in terms of column
$y = x_1a_1 + x_2a_2 +...+ x_na_n$\\
\textbf{General Examples}\\
\textbullet Picking out columns and rows An important identity is $Ae_j = a_j$, the jth column of A. (In other words, $(A^T e_i)^T$ is the ith row of A.)\\
\textbullet \textit{Summing or  averaging columns or rows}:\\ The m-vector A1 is the sum of the columns of A; its ith entry is the sum of the entries in the ith row of A. The m-vector $A(\textbf{1}/n)$ is the average of the columns of A; its ith entry is the average of the entries in the ith row of A. In a similar way, $A^T \textbf{1}$ is an n-vector, whose jth entry is the sum of the entries in the jth column of A.\\

\subsection{Complexity}
\textit{addition}: $mn$\\
\textit{sparse matrix addition}: If A or B or both are sparse $min\{nnz(A), nnz(B)\}$\\
\textit{vector multiplication $A_{mxn}$ with n-vector}: $m(2n-1) \approx 2mn$\\
\textit{Matrix Transpose}: 0 flops